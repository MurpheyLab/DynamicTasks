\documentclass[11pt]{article}
\usepackage[margin=1.25in]{geometry}
\usepackage{graphicx}
\usepackage{gensymb}
\usepackage{multirow}
\usepackage{enumitem}

\begin{document}
\title{Subject Data Sheet}

\maketitle
\section*{Subject Information}
\begin{itemize}[leftmargin=2in]
\item[Subject Code:]
\item[Age:]
\item[Gender:]
\item[Date of Stroke:]
\item[Affected Limb (R vs.\ L):]
\item[Dominant Limb (R vs.\ L):]
\item[Tested Limb (R vs.\ L):]
\item[Arm length:]
\item[Forearm length:]

\end{itemize}
\section*{Measurements}

\begin{table}[!h]
\centering
\begin{tabular}{c | c | c | c | c}
\textbf{Paretic Arm} &\textbf{Measurement 1} & \textbf{Measurement 2} & \textbf{Measurement 3} & \textbf{Mean} \\
\hline
\multirow{2}{*}{\textbf{Arm Weight}}& & & &\\
& & &\\
\hline
\multirow{2}{*}{\textbf{Max Isometric Force}}& & & &\\
& & &
\end{tabular}
\end{table}


\begin{table}[!h]
\centering
\begin{tabular}{c | c | c | c | c}
\textbf{Nonparetic Arm} &\textbf{Measurement 1} & \textbf{Measurement 2} & \textbf{Measurement 3} & \textbf{Mean} \\
\hline
\multirow{2}{*}{\textbf{Arm Weight}}& & & &\\
& & &\\
\hline
\multirow{2}{*}{\textbf{Max Isometric Force}}& & & &\\
& & &
\end{tabular}
\end{table}

\section*{Biodex Position for Measuring Abduction}
\begin{itemize}[leftmargin=2in]
\item[Robot Position:]
\item[Chair Height (80 degrees abduction):]
\end{itemize}

\section*{Biodex Home Position}
\begin{itemize}[leftmargin=2in]
\item[Robot Position (home):]
\item[Chair Height (70 degrees abduction):]
\item[Chair Position (Right Arm):]
\item[Chair Position (Left Arm):]
\end{itemize}

\section*{Chair Position}
\begin{itemize}[leftmargin=2in]
\item[Chair Position (Right Arm):]
\item[Chair Position (Left Arm):]
\end{itemize}

\section*{General Observations:}
\end{document}
